\documentclass[12pt, a4paper]{article}

% Geometry and spacing
\usepackage[margin=0.7in]{geometry}
\usepackage{setspace}
\setstretch{1.5}

% Fonts and language settings
\usepackage{fontspec}
\usepackage{xeCJK}
\setmainfont{Times New Roman}
\setCJKmainfont{SimSun} % 设置中文字体为宋体(SimSun)

% Math and graphics
\usepackage{amsmath}
\usepackage{graphicx}
\usepackage{float}

% Hyperlinks
\usepackage{hyperref}
\hypersetup{
    colorlinks=true,
    linkcolor=blue,
    urlcolor=cyan,
}

% Section titles spacing
\usepackage{titlesec}
\titlespacing*{\section}{0pt}{*1}{*1}
\titlespacing*{\subsection}{0pt}{*1}{*1}


\title{Lecture 13: Money and Prices in the Long Run}
\author{}
\date{}

\begin{document}
\maketitle

\section{货币制度}

\subsection{货币的含义与职能}
\begin{itemize}
    \item 货币:经济中人们经常用于向他人购买物品与服务的一组资产。
    \item 货币的职能:
    \begin{itemize}
        \item 交换媒介:买者在购买物品与劳务时给予卖者的东西。
        \item 计价单位:人们用来表示价格和记录债务的标准。
        \item 价值储藏手段:把现在的购买力转变为未来的购买力。
    \end{itemize}
\end{itemize}

\subsection{货币供给}
\begin{itemize}
    \item 货币供给:经济中流通的货币量。
    \item 组成部分:
    \begin{itemize}
        \item 通货:公众手中持有的纸币钞票和铸币。
        \item 活期存款:储户可以随时支取的银行账户余额。
    \end{itemize}
\end{itemize}

\subsection{银行与货币供给}
\begin{itemize}
    \item 部分准备金银行制度:银行将部分存款作为准备金,其他用于贷款。
    \item 准备金率\( R \)等于准备金占所有存款的比例。
    \item 货币乘数:银行体系用1美元准备金所产生的货币量,货币乘数等于 \( \frac{1}{R} \)。
\end{itemize}


\subsection{央行控制货币供给的三种工具}

\subsubsection{公开市场操作}
\begin{itemize}
    \item \textbf{定义}:央行通过买卖政府债券来调节货币供给。
    \item \textbf{增加货币供给}:
    \begin{itemize}
        \item 央行购买政府债券。
        \item 购买政府债券时,央行向市场注入货币。
        \item 这些新增的货币存入银行,增加银行准备金。
        \item 银行用新增的准备金发放更多贷款,进一步增加货币供给。
    \end{itemize}
    \item \textbf{减少货币供给}:
    \begin{itemize}
        \item 央行出售政府债券。
        \item 出售政府债券时,央行从市场收回货币,减少流通中的货币量。
    \end{itemize}
\end{itemize}

\subsubsection{法定准备金}
\begin{itemize}
    \item \textbf{定义}:央行规定银行必须持有的准备金比例。
    \item \textbf{增加货币供给}:
    \begin{itemize}
        \item 央行降低法定准备金比例。
        \item 银行可以用每一元准备金发放更多的贷款。
        \item 增加货币乘数与货币供给。
    \end{itemize}
    \item \textbf{减少货币供给}:
    \begin{itemize}
        \item 央行提高法定准备金比例。
        \item 银行必须持有更多的准备金,减少可用于贷款的资金。
        \item 减少货币乘数与货币供给。
    \end{itemize}
    \item \textbf{注意}:
    \begin{itemize}
        \item 央行很少频繁改变法定准备金,因为频繁改变会干扰银行业务。
    \end{itemize}
\end{itemize}

\subsubsection{贴现率}
\begin{itemize}
    \item \textbf{定义}:央行向商业银行发放贷款的利率。
    \item \textbf{增加货币供给}:
    \begin{itemize}
        \item 央行降低贴现率。
        \item 鼓励银行向央行借入更多的贷款。
        \item 银行获得更多准备金,可以发放更多贷款,增加货币供给。
    \end{itemize}
    \item \textbf{减少货币供给}:
    \begin{itemize}
        \item 央行提高贴现率。
        \item 银行借款成本上升,减少向央行借款。
        \item 银行准备金减少,减少贷款发放,降低货币供给。
    \end{itemize}
    \item \textbf{注意}:
    \begin{itemize}
        \item 在没有金融危机的情况下,央行很少使用贴现贷款,因为央行是“最后贷款人”。
    \end{itemize}
\end{itemize}

\subsubsection{联邦基金利率}
\subsection*{定义}
联邦基金利率是指银行之间进行隔夜贷款时的利率。在一天结束时,如果银行发现自己的准备金不足,可以向其他有超额准备金的银行借款,这种贷款的利率就是联邦基金利率。在中国,这种利率被称为同业拆借利率或隔夜拆借利率。

\subsection*{期限}
虽然称为隔夜拆借利率,但实际上这种贷款可以有不同的期限,如1天、3天、6天、1个月、3个月和6个月等。但一般来说,贷款期限不会超过6个月。由于时间都很短,所以被统称为隔夜拆借。

\subsection*{央行的作用}
央行通过公开市场操作来控制联邦基金利率。具体做法如下:
\begin{itemize}
    \item \textbf{降低联邦基金利率}:央行购买政府债券,向市场注入货币,增加银行体系中的准备金。当准备金供给增加时,银行之间借款的需求减少,从而降低联邦基金利率。
    \item \textbf{提高联邦基金利率}:央行出售政府债券,从市场回收货币,减少银行体系中的准备金。当准备金供给减少时,银行之间借款的需求增加,从而提高联邦基金利率。
\end{itemize}

\subsection*{联邦基金利率的重要性}
联邦基金利率是一个重要的经济指标,因为它会影响许多其他利率,如商业贷款利率、消费贷款利率等。许多利率互相之间是高度关联的,因此联邦基金利率的改变会导致其他利率的改变,并对整个经济产生重大影响。

\subsection*{快问快答}
\begin{itemize}
    \item \textbf{联邦基金利率与贴现率有什么不同?}
    \begin{itemize}
        \item 联邦基金利率是银行之间进行隔夜贷款的利率。
        \item 贴现率是央行向商业银行发放贷款的利率。
    \end{itemize}
    \item \textbf{央行如何通过公开市场操作钉住其确定的联邦基金利率?}
    \begin{itemize}
        \item 央行通过买卖政府债券来调节银行体系中的准备金供给,从而影响联邦基金利率。
        \item 例如,为使联邦基金利率上升,央行会出售政府债券,减少银行体系中的准备金供给,从而增加银行之间的借款需求,导致联邦基金利率上升。
    \end{itemize}
\end{itemize}

\subsection*{图示说明}
假设央行希望提高联邦基金利率,过程如下:
\begin{itemize}
    \item 央行出售政府债券。
    \item 银行体系中的准备金减少,联邦基金的供给减少。
    \item 导致联邦基金利率上升。

% \begin{figure}[H]
%     \centering
%     \includegraphics[width=0.8\textwidth]{fed_funds_rate.png}
%     \caption{联邦基金市场}
% \end{figure}

在图中,供给曲线 \( S \) 从 \( S1 \) 移动到 \( S2 \),使得均衡利率从 \( F1 \) 上升到 \( F2 \)。
\end{itemize}


\section{货币增长与通货膨胀}

\subsection{货币数量论}
\begin{itemize}
    \item 货币数量论:认为货币数量决定货币价值。
    \item 数量方程式: \( M \times V = P \times Y \)
    \begin{itemize}
        \item \( M \) 为货币供给。
        \item \( V \) 为货币流通速度。
        \item \( P \) 为物价水平。
        \item \( Y \) 为真实GDP。
    \end{itemize}
\end{itemize}

\subsection{古典二分法和货币中性}
\begin{itemize}
    \item 古典二分法:名义变量和真实变量的理论区分。
    \item 货币中性:货币的变动不影响真实变量。
\end{itemize}

\subsection{通货膨胀的成本}
\begin{itemize}
    \item 皮鞋成本:人们减少货币持有量时所浪费的资源。
    \item 菜单成本:改变价格的成本。
    \item 相对价格波动与资源配置不当:通货膨胀扭曲相对价格。
    \item 税收扭曲:通货膨胀使名义收入比真实收入增长得更快。
    \item 混乱与不方便:通货膨胀改变了衡量经济交易的尺度。
    \item 任意的财富再分配:高于预期的通货膨胀将购买力从债权人向债务人转移。
\end{itemize}

\section{课堂练习}
\begin{enumerate}
    \item 情形1: 你打扫房间时在沙发坐垫下找到一张50美元的钞票。你把钱存入活期储蓄账户,央行规定的法定存款准备金率是20\%。
    \begin{itemize}
        \item 货币供给量增加的最大数量是多少 \[ 50 \times \frac{1}{0.2} = 250, 250 - 50 = 200 \]
        \item 货币供给量增加的最小数量是多少?\( 50 - 50 = 0 \)
    \end{itemize}

    \item 情形2: 计算2009年的名义GDP和P,并计算从2008年到2009年的通货膨胀率(价格上涨的比例)。假设技术进步使2009年的产出Y增加到824,计算从2008年到2009年的通货膨胀率。
    \begin{itemize}
        \item 已知:2008年 \( Y = 800 \), \( V \) 不变, \( MS = \$2000 \) ,\( P = \$5 \)
        \item 2009年,央行货币供给增加5\%,增加到2100美元
    \end{itemize}

    \item 你在银行存款1000美元,期限为一年。
    \begin{itemize}
        \item 情形 1:  通货膨胀率 = 0\%,名义利率 = 10\%
        \item 情形 2:  通货膨胀率 = 10\%,名义利率 = 20\%
        \begin{enumerate}
            \item 在哪种情形中,你存款的真实价值增长的更快?
            \item 假定税率为25\%,在哪种情形中,你纳税最多?
            \item 计算税后名义利率,然后减去通货膨胀率计算两种情形的税后真实利率。
        \end{enumerate}
    \end{itemize}
\end{enumerate}

\end{document}
