\documentclass{article}
\usepackage{ctex}
\usepackage{booktabs}
\usepackage[margin=1in]{geometry}
\setCJKmainfont{SimSun} % 设置中文字体为宋体

\title{Half Marathon Plan}
\author{}
\date{}

\begin{document}
\maketitle

\begin{table}[h!]
\centering
\begin{tabular}{@{}cccccccc@{}}
\toprule
周次 & 周一    & 周二    & 周三    & 周四       & 周五       & 周六        & 周日        \\ 
\midrule
1    & 跑休/XT & 3.2公里 & 跑休/XT & 6.4公里    & 跑休/XT    & 3.2公里     & 8公里LSD  \\
2    & 跑休/XT & 3.2公里 & 跑休/XT & 8公里      & 跑休/XT    & 9.6公里LSD  & 24公里     \\
3    & 跑休/XT & 3.2公里 & 跑休/XT & 慢跑+HMP   & 跑休/XT    & 11.2公里LSD & 25.6公里   \\
4    & 跑休/XT & 3.2公里 & 跑休/XT & 慢跑+HMP   & 跑休/XT    & 11.2公里LSD & 27.2公里   \\
5    & 跑休/XT & 6.4公里 & 跑休/XT & 慢跑+HMP   & 跑休/XT    & 6.4公里LSD  & 22.4公里   \\
6    & 跑休/XT & 3.2公里 & 跑休/XT & 慢跑+HMP   & 跑休/XT    & 12.8公里LSD & 32公里     \\
7    & 跑休/XT & 3.2公里 & 跑休/XT & 慢跑+HMP   & 跑休/XT    & 14.4公里LSD & 33.6公里   \\
8    & 跑休/XT & 3.2公里 & 跑休/XT & 慢跑+HMP   & 跑休/XT    & 16公里LSD   & 35.2公里   \\
9    & 跑休/XT & 3.2公里 & 跑休/XT & 慢跑+HMP   & 跑休/XT    & 14.4公里LSD & 36.8公里   \\
10   & 跑休/XT & 3.2公里 & 跑休/XT & 慢跑+HMP   & 跑休/XT    & 比赛日      & 35.4公里   \\ 
\bottomrule
\end{tabular}
\caption{初级半程马拉松训练计划}
\end{table}
\vspace{3mm}
\section{术语解释}
\subsection{配速:跑一公里所用的时间。}
\subsection{跑休/XT:休息一天,或进行适量的不会对身体造成冲击的交叉训练,如瑜伽或游泳。}
\subsection{LSD:长距离慢跑,其目的是锻炼耐力,应以舒适的、可对话的配速进行,即比目标比赛配速慢37.5~75秒。}
\subsection{HMP:半程马拉松配速,即你希望在比赛中保持的配速。分别以1.6公里轻松跑作为热身和冷身。}


\end{document}
